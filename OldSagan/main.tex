\documentclass{amsart}
\usepackage{graphicx} % Required for inserting images

\title{Amateurish pi stunt notes}
\author{Permutation enthusiasts}
\date{November 2024}

\DeclareMathOperator{\inv}{inv}
\DeclareMathOperator{\sgn}{sgn}
\DeclareMathOperator{\Tr}{Tr}
\newcommand{\ZZ}{\mathbb{Z}}
\newcommand{\CC}{\mathbb{C}}
\renewcommand{\S}{\mathcal{S}}
\renewcommand{\H}{\mathcal{H}}


\begin{document}

\maketitle

\section{Problems 1.13}

\begin{enumerate}
    \item[1]
    \begin{enumerate}
        \item  Consider the transposition $\pi = (i,j)$.  It swaps the order of exactly $2(j-i-1) + 1$ pairs of inputs.  This is because there are $j-i-1$ entries strictly between $i$ and $j$, and $\pi$ swaps both $i$ and $j$ with all of these.  It also swaps the order of $i$ and $j$.  Therefore, $\inv((i,j)) \equiv 1 \bmod 2$.  Next we show that $\inv$ is a homomorphism to $\ZZ/2$.  To do this, let $\inv_{i,j}(\pi)$ be $1$ if $\pi$ swaps the order of $i$ and $j$ and $0$ otherwise, so $\inv(\pi) = \sum_{i \ne j}\inv_{i,j}(\pi)$.  If we have two permutations $\pi$ and $\tau$, we have $\inv_{i,j}(\tau\pi) = \inv_{i,j}(\pi) + \inv_{\pi(i),\pi(j)}(\tau) \bmod 2$. I.e. $i$ and $j$ get swapped if $\pi$ swaps $i$ and $j$ or $\tau$ swaps $\pi(i), \pi(j)$ (but not both because they get swapped back if so).
        Therefore
        \begin{align*}
        \inv(\tau\pi) &= \sum_{i\ne j} \inv_{i,j}(\tau\pi) \\
                      &\equiv \sum_{i\ne j} \inv_{i,j}(\pi) + \inv_{\pi(i),\pi(j)}(\tau) \bmod 2 \\
                      &\equiv \left(\sum_{i \ne j} \inv_{i,j}(\pi) + \sum_{i\ne j}\inv_{\pi(i),\pi(j)}(\tau)\right) \bmod 2  \\
                      &\equiv \inv(\pi) + \inv(\tau) \bmod 2
        \end{align*}
        Taken together, these facts imply that if $\pi$ is a product of $k$ transpositions, then $\inv(\pi) \equiv k \bmod 2$.
        \item We defined $\sgn(\pi) = (-1)^k$ when $\pi$ is a product of $k$ transpositions.  It may be that $\pi$ can be written as a product of $k$ transpositions and also as a product of $\ell\ne k$ transpositions.  However, we can see that $\sgn(\pi)$ is well defined because $\inv(\pi)$ is well defined, and we saw in (a) that $k \equiv \inv(\pi) \equiv \ell \bmod 2$.  Thus $(-1)^k = (-1)^\ell$.
    \end{enumerate}
    \item[2]
    \begin{enumerate}
        \item We clearly have $\epsilon \in G_s$ because $\epsilon s = s$ by an axiom of group actions.  To see that $G_s$ is closed under multiplication, let us be given $g,h \in G_s$, and we compute $(gh)s = g(hs) = gs = s$ (the second equality uses a group action axiom).
        \item Define $\phi : G/G_s \to \mathcal{O}_s$ by $\phi(hG_s) = hs$.  This is well-defined because if $h=hg$ for $g \in G_s$, then $(hg)s = h(g(s)) = hs$.  (I'm assuming left cosets.)  The map $\phi$ is surjective because if we are given any $h \in G$, then $\phi(hG_s) = hs$.  The map $\phi$ is injective because if $\phi(hG_s) = \phi(kG_s)$, then $hs = ks$, so $k^{-1}h \in G_s$ and hence $hG_s = kG_s$.
        \item We know $\left| G/G_s\right| = |G|/|G_s|$ by some isomorphism theorem, and by (b) we know $|\mathcal{O}_s| = \left| G/G_s\right|$.
    \end{enumerate}
    \item[3]
    \begin{enumerate}
        \item We must show that every matrix $X(\pi)$ has exactly one $1$
        in each row and column, and zeros elsewhere.  The definition of the permutation representation has $X(\pi)_{i,j} = \delta_{i=\pi(j)}$.
        Because $\pi$ is a permutation, for each $i$ there is exactly one $j$ such that $i=\pi(j)$, and for each $j$ there is exactly one $i$ such that $i=\pi(j)$.
        \item A fixed point $i$ of $\pi$ has $\pi(i) = i$, so $X(\pi)_{i,i}=1$, so there is a $1$ on the diagonal of $X(\pi)$
        in position $(i,i)$ exactly when $i$ is fixed under $\pi$.  Thus $\Tr(X(\pi))$ is the number of fixed points of $\pi$.
    \end{enumerate}
    \item[4] Since $G$ is finite, it can be written as $\oplus_i C_{j_i}$, where $C_{j_i}$ is a cyclic group of order $j_i$, say with generator $g_i$.  By Corollary 1.6.8 (see also Problem 12), we must have $X(g) = cI$ for all $g \in G$, so $X$ is one dimensional, and $X(g_i)$ is some $j_i$th root of unity.  All such representations are irreducible since they are one dimensional.
    \item[5]
    \begin{enumerate}
        \item Let $g \in N$ and $h \in G$.  Then $X(hgh^{-1}) = X(h)X(g)X(h^{-1}) = X(h)X(h)^{-1} = I$, so $hgh^{-1} \in N$.  This holds for all $g \in N$, so $N$ is normal.  A condition is: $X$ is faithful iff $N = \{\epsilon\}$.  To see one direction, suppose $X$ is faithful.  Then $I = X(\epsilon) = X(g)$ only if $g = \epsilon$; hence $N = \{\epsilon\}$.  For the other direction, suppose $N = \{\epsilon\}$ and $X(g) = X(h)$ for some $g,h$.  Then $X(gh^{-1}) = I$, so $gh^{-1} = \epsilon$, so $g=h$, and $X$ is faithful.
        \item One direction is immediate, because if $g \in N$, then $\Tr(X(g)) = \Tr(I) = d$.  For the other direction, suppose $\chi(g) = d$.  TODO
        \item For one direction, suppose that $h \in \bigcap_i g_iHg_i^{-1}$.  Then for all $i$, we have $hg_i \in g_iH$, so $hg_iH \subseteq g_iHH = g_iH$ sends each coset to itself, so $h \in N$.  Conversely, if $X(h) = I$, then $h$ sends each coset to itself, so $hg_iH \in g_iH$ for all $i$.  Hence $h \in g_iHg_i^{-1}$ for all $i$.
        \item
        \begin{enumerate}
        \item Trivial: this is faithful exactly if $G$ is trivial
        \item Regular: always
        \item Coset: when the intersection of the conjugates of $H$ is trivial (see previous)
        \item Sign for $\S_n$: for $\S_1$ and $\S_2$
        \item Defining for $\S_n$: always
        \item Degree 1 for $C_n$: exactly when $X(g)$ is a primitive root of unity, for a generator $g$
        \end{enumerate}
        \item 
          \begin{enumerate}
              \item $Y$ is well-defined because if $gN = hN$, then there is $n \in N$ so that $gn = h$.  Then $X(h) = X(gn) = X(g)X(n) = X(g)I = X(g)$.  It is faithful because if $Y(gN) = I$, then by definition we have $X(g) = I$, so $g \in N$, ie. the only coset that maps to $I$ under $Y$ is $\epsilon N$.
              \item Whether or not a representation is irreducible depends only on the set of matrices (or endomorphisms) in the image.  The image of $Y$ is the same as the image of $X$.  Said another way, if $X(g)(V) \subseteq V$ for some subspace $V$ and for all $g$, then $Y(gN)(V) = X(g)(V) \subseteq V$ as well, and vice versa.
              \item The representation $Y$ is the regular representation of $G/H$.  To see this, let us start by finding the kernel $N$ of the coset representation.  Suppose $n \in N$, so $ngH = gH$ for all $g$.  Because $H$ is normal, we have $ngH = Hng$, so $Hng = gH$.  Thus $Hn = gHg^{-1} = H$, so $n \in H$.  The entire argument runs backward, so $N=H$.  Let $V$ be the coset representation (so $Y$ is a map $Y:G/H \to GL(V)$) and consider the map $\theta : V \to \mathbb{C}[G/H]$ defined by $\theta(gH) = gH$.  This is clearly a bijection, so we just need to check it is a $G/H$-homomorphism.  To see this, we compute $\theta(Y(gH)(hH)) = ghH$ and $gH\theta(hH) = ghH$ (by the definition of group multiplication in $G/H$).
          \end{enumerate}
    \end{enumerate}
    \item[(6)]
    \begin{enumerate}
        \item To see that $X$ is a representation, we just need to check that $X(gh) = X(g)X(h)$.  We compute $X(gh) = Y(ghN) = Y(gNhN) = Y(gN)Y(hN) = X(g)X(h)$, where in the middle we used the multiplication in $G/H$.
        \item Let $g \ker(X)$, so $I = X(g) = Y(gN)$.  Since $Y$ is faithful, we have $gN = \epsilon N$, so $g \in N$.  Conversely, any $g in N$ is in $\ker(X)$ because $X(g) = Y(gN) = Y(\epsilon N) = I$.
        \item This is the same as (5)(e)(ii) -- the irreducibility only depends on the image set of matrices, which remains the same under lifting.
    \end{enumerate}
    \item[(7)] The block decomposition of $X$ expresses $V$ as the internal direct sum $W + Y$, where if we write any vector $(w,y)$ aligned with the block form, we have $X(g)(w,y) = (A(g)w + B(g)y, C(g)y)$.  The quotient map $V \to V/W$ projects to the second coordinate and is a $G$-homomorphism which takes the action $X$ to $C$.  Maschke's theorem says that $V$ is isomorphic to a block diagonal action with the matrices $A$ and $C$, and this is exactly the actions on $W$ and $V/W$.
    \item[(8)] I'm not sure about these
    \begin{enumerate}
        \item The action of $G$ can be given by a matrix in the basis?
        \item The map $\theta$ is linear and for all $g \in G$ and $b$ in the basis, we have $g\theta(b) = \theta (gb)$?
        \item For all $b,c$ in the basis, we have $\langle b,c \rangle = \langle gb,gc \rangle$?
    \end{enumerate}
    \item[(10)] The map $X(r) = \begin{bmatrix} 1 & \log r \\ 0 & 1\end{bmatrix}$ satisfies 
    \begin{align*}
        X(r)X(s) & = \begin{bmatrix} 1 & \log r \\ 0 & 1\end{bmatrix}\begin{bmatrix} 1 & \log s \\ 0 & 1\end{bmatrix} \\
        & = \begin{bmatrix} 1 & \log(rs) \\ 0 & 1\end{bmatrix} \\
        & = X(rs),
    \end{align*}
    and we can see $X(r)\begin{bmatrix} c \\ 0 \end{bmatrix} = \begin{bmatrix} c \\ 0 \end{bmatrix}$.
    \item[(11)] Let $H = \S_{n-1} \subseteq \S_n$, and let $S$ be
    the set of tabloids of shape $(n-1, 1)$.  We want to show that $\CC\H \cong \CC\mathbf{S} \cong \CC\{\mathbf{1}, \dots, \mathbf{n}\}$.  First we need to find a transversal for $H$.  Note that $|H| = (n-1)!$, so the index of $H$ in $\S_n$ is $n$, so it suffices to show our $n$ chosen cosets are pairwise disjoint.  To do this, suppose $(i,n)H = (j,n)H$, so $(j,n)(i,n) \in H$.  Note this product fixes $n$ (equivalently, is in $H$) exactly when $i = j$.  Hence the cosets are disjoint.  It is also useful to observe that if $i \ne j$, we have $(i,n)(j,n) = (j,n)(i,j)$, so $(i,n)\mathbf{(j,n)H} = \mathbf{(j,n)H}$, while if $i=j$, then we compute that $(i,n)$ interchanges the cosets $\epsilon H$ and $(i,n)H$.  This gives us the action on cosets.
    
    Now define the equivalence $\theta: \CC\H \to \CC\{\mathbf{1}, \dots, \mathbf{n}\}$ by $\theta(\mathbf{(i,n)H}) = \mathbf{i}$ and $\iota:\CC\mathbf{S} \to \CC\{\mathbf{1}, \dots, \mathbf{n}\}$ by taking a tabloid basis element to the basis element of $\CC\{\mathbf{1}, \dots, \mathbf{n}\}$ associated with the single entry in the second level of the tabloid.  Since $\{(i,n)\}_{i=1}^n$ generates $\S_n$, it suffices to show that these maps commute with the action of these involutions.  This is immediate for $\iota$ because the image is by definition the entry in the bottom of the tabloid.  For $\theta$, we use the coset action we determined above, so $\theta((i,n)\mathbf{(j,n)H}) = \theta(\mathbf{(j,n)H}) = \mathbf{j}$ if $i\ne j$, and $\theta((i,n)\mathbf{(i,n)H}) = \mathbf{n}$ and $\theta((i,n)\mathbf{\epsilon H}) = \mathbf{i}$.  That is, the action of $(i,n)$ swaps the corresponding pairs of basis elements on both sides of $\theta$.
    \item[(12)] By Corollary 1.6.8, any matrix that commutes with $X(g)$ for all $g$ must be of the form $cI$.  If $g \in Z_G$, then by definition $X(g)$ commutes with $X(h)$ for all $h \in G$, and the conclusion is immediate.
    \item[(13)] Let $G$ be the abelian group formed by the matrices $X_i$.  So the map $Y(X_i) = X_i$ is a $d$-dimensional representation of $G$.  By Maschke's theorem, there is a single matrix $T$ such that $TX_iT^{-1}$ is a block diagonal matrix of irreducible representations.  It remains to show that any irreducible representation of an abelian group is 1-dimensional, which maybe we just know that, or maybe we observe that by Corollary 1.6.8, if $G$ is an abelian group, any image matrix in an irreducible representation must be a multiple of the identity and thus must be 1-dimensional.
    \item[(14)] Suppose towards a contradiction that $X$ is reducible,
    so up to isomorphism we can simultaneously write the matrices $X(g)$
    in a nontrivial block form.  But then $X(g)$ commutes with block diagonal matrices with blocks $xI$, $yI$, for any $x,y \in \CC$.  Many such matrices are not of the form $cI$, which is a contradiction.
    \item[(15)]
    \begin{enumerate}
    \item We must check that $(X \hat{\otimes} Y)(gh) = (X \hat{\otimes} Y)(g) (X \hat{\otimes} Y)(h)$.  To do this, we compute:
    \begin{align*}
    (X \hat{\otimes} Y)(g) (X \hat{\otimes} Y)(h) &= (X(g) \otimes Y(g))(X(h) \otimes Y(h)) \\
    &= X(g)X(h) \otimes Y(g)Y(h) \\
    & = X(gh) \otimes Y(gh) \\
    &= (X \hat{\otimes} Y)(gh)
    \end{align*}
    The second equality uses Lemma 1.7.7.
    \item We can compute
    \begin{align*}
        (\chi\hat{\otimes}\psi)(g) &= \Tr((X\hat{\otimes}Y)(g)) \\
                                   &= \Tr(X(g) \otimes Y(g)) \\
                                   &= \sum_{i}X(g)_{i,i}\Tr(Y(g)) \\
                                   &= \sum_iX(g)_{i,i} \psi(g) \\
                                   &= \chi(g)\psi(g)
    \end{align*}
    \item If $X$ and $Y$ are both the irreducible 2-dimensional representation of $\S_3$, then $X \hat{\otimes} Y$ has dimension $4$, but $\S_3$ has no $4$-dimensional irreducible representations.
    \item We can check that it is irreducible by computing
    \begin{align*}
        \langle \chi\hat{\otimes}\psi, \chi\hat{\otimes}\psi\rangle &=
         \frac{1}{|G|}\sum_g (\chi\hat{\otimes}\psi)(g) (\chi\hat{\otimes}\psi)(g^{-1}) \\
         &= \frac{1}{|G|}\sum_g \chi(g)\psi(g) \chi(g^{-1})\psi(g^{-1})\\
         &= \frac{1}{|G|}\sum_g \psi(g)\psi(g^{-1})\\
         &= \langle \psi, \psi \rangle \\
         &= 1
    \end{align*}
    This relies on the fact that $X$ is one-dimensional, so $\Tr(X(g^{-1})) = 1/\Tr(X(g))$, so $\chi(g)\chi(g^{-1}) = 1$.
    \end{enumerate}
    \item[(16)] There are five cycle types/conjugacy classes  $\epsilon, (1,2), (1,2,3), (1,2)(3,4), (1,2,3,4)$, of sizes 1, 6, 8, 3, and 6, respectively.
     Because there are five, we are expecting five irreducible representations.  We know the trivial $\chi^{(1)}$ and sign $\chi^{(2)}$ representations are irreducible, and we know the representation $\chi^{\perp}$ orthogonal to the trivial one inside the defining representation, which we can verify is irreducible by computing its self inner product.  In addition, we compute
     the character for $\chi^{(2)} \hat{\oplus} \chi^{\perp}$ and see it too
     is irreducible.

     For the final irreducible, consider the normal subgroup $N$ which is
     $\epsilon$ and the conjugacy class of $(1,2)(3,4)$ (This is the
     Klein four group $\ZZ/2 \times \ZZ/2$).  In fact, we have
     $\S_4 / N \cong \S_3$.  To see this, consider the map $\phi:\S_3 \to \S_4$
     defined by $\phi(\pi) = \pi N$.  If $\pi N = \rho N$, then there is $n \in N$ so that $\pi = \rho n$.  If $n \ne \epsilon$, then note that $n$, and thus $\rho n$, does not fix 4 (here $\rho \in \S_3$, so if $n$ permutes $4$ away from itself, $\rho$ cannot put it back).  On the other hand $\pi$ does fix 4.  This contradiction implies that $n = \epsilon$, so $\phi$ is injective.  Since $|\S_3| = 6$ and $|\S_4/N| = 6$, in fact $\phi$ is an isomorphism.  To compute the quotient map from $\S_4$ to $\S_3$ on a permutation $\pi$ which does not fix $4$, we need to find $\rho n = \pi$, with $n \in N$ and $\rho$ fixing 4.

     So we can use the lifting process from problem 6 to lift each of the 3 irreducible representations of $\S_3$ to representations of $\S_4$.
     The trivial and sign representations lift to the trivial and sign representations, respectively, and give us nothing new.  But the third irreducible $\chi^{(3)}$ does give us the final, 2 dimensional, irreducible representation $\chi^{(3)}$ of $\S_4$.  (We can check that it is irreducible by computing its self inner product.)
    \begin{center}
        \begin{tabular}{r|ccccc}
                    & $\epsilon$ & $(1,2)$ & $(1,2,3)$ & $(1,2)(3,4)$ & $(1,2,3,4)$ \\
\hline
$\chi^{(1)}$ trivial  &     1    &     1   &      1    &      1       &     1    \\
 $\chi^{(2)}$ sign  &     1      &    -1   &    1      &      1       &     -1 \\
    $\chi^{\perp}$  &     3      &   1     &    0      &     -1       &    -1   \\
    $\chi^{(2)} \hat{\otimes} \chi^{\perp}$ &  3  & -1 & 0 & -1 & 1 \\
    $\chi^{(3)}$    & 2  & 0 & -1 & 2 & 0 \\
        \end{tabular}
    \end{center}
    \item[(17)]  
    \begin{enumerate}
        \item We can flip ($\tau$, order 2) rotate ($\rho$, order $n$), and playing with a shape shows that $\rho \tau = \tau \rho^{-1}$.
        \item If we have any sequence of $\tau$ and $\rho$, we can slide all $\rho$ to the right using the relation $\rho \tau = \tau \rho^{-1}$
        \item We compute:
        \[
        (\tau^e \rho^\ell)\rho^j(\rho^{-\ell}\tau^e) =
        \left\{ \begin{array}{ll} \rho^j & \textnormal{if $e=0$} \\
                                  \rho^{-j} & \textnormal{if $e=1$}
                                \end{array}\right.
        \]
        \[
        (\tau^e \rho^\ell)\tau\rho^j(\rho^{-\ell}\tau^e) =
        \left\{ \begin{array}{ll} \tau\rho^{2\ell-j} & \textnormal{if $e=0$} \\
                                  \tau\rho^{j-2\ell} & \textnormal{if $e=1$}
                                \end{array}\right.
        \]
        These relations determine the conjugacy classes of $D_n$.  The answer depends on whether $n$ is odd (the issue is whether $2$ is relatively prime to $n$, ie. whether $2$ is a generator of the additive group $\ZZ/n$).

        If $n$ is even, then the conjugacy classes are
        \[
        \{\epsilon\}, \{\rho^1, \rho^{n-1}\}, \dots, \{\rho^{n/2-1}, \rho^{n/2+1}\}, \{\rho^{n/2}\}, \{\tau, \tau\rho^2, \dots, \tau\rho^{n-2}\}, \{\tau\rho, \tau\rho^3, \dots, \tau\rho^{n-1}\},
        \]
        so there are $n/2 + 3$ classes total. If $n$ is odd, the conjugacy classes are
        \[
        \{\epsilon\}, \{\rho^1, \rho^{n-1}\}, \dots, \{\rho^{\frac{n-1}{2}}, \rho^{\frac{n+1}{2}}\},  \{\tau, \tau\rho, \dots, \tau\rho^{n-1}\},
        \]
        so there are $\lfloor n/2 \rfloor + 2$ classes total.
        \item There are some simple-to-define representations $X_j$, which we will
        check are irreducible.  Define $X_j$ as follows, where $\rho$ is mapped
        to a rotation
        \[
        X_j(\rho) = \begin{bmatrix} \cos \frac{2\pi j}{n} & -\sin \frac{2\pi j}{n} \\
                                    \sin \frac{2\pi j}{n} & \cos \frac{2\pi j}{n}
                    \end{bmatrix}
        \]
        and $\tau$ to a flip
        \[
        X_j(\tau) = \begin{bmatrix} 1 & 0 \\ 0 & -1 \end{bmatrix}.
        \]
        The representation $X_1$ is the ``defining'' representation of $D_n$ as it's
        usually defined acting on a $n$-sided polygon in the plane. (Although this is a 2-dimensional \emph{complex} representation!)

        Let $\chi_j$ be the character of $X_j$.  Note that 
        \[
        \chi_j(\rho^i) = \Tr(X_j(\rho^i)) = 2\cos \frac{2\pi i j}{n} \quad \textnormal{and} \quad \chi_j(\tau\rho^i) = 0.
       \]
       In particular, we can ignore $\tau\rho^i$ for all subsequent calculations of characters.

       We can compute
       \begin{align*}
           \langle \chi_j, \chi_j \rangle &= \frac{1}{|D_n|} \sum_{i=0}^{n-1} \chi_j(\rho^i) \chi_j(\rho^{-i}) \\
           &= \frac{1}{2n} \sum_{i=0}^{n-1} \left(2\cos\frac{2\pi i j}{n}\right)\left(2\cos \frac{-2\pi i j}{n}\right)\\
           &= \frac{2}{n} \sum_{i=0}^{n-1} \cos^2 \left(\frac{2\pi i j}{n}\right)\\
           &= \frac{2}{n} \sum_{i=0}^{n-1} \frac{1+\cos\frac{4\pi i j}{n}}{2} \\
           &= 1 + \frac{1}{n}\sum_{i=0}^{n-1} \cos\frac{4\pi i j}{n}\\
           &= 1 + \delta_{j=0}
       \end{align*}
       The last equality uses the fact that the sum is zero as long as $j \ne 0$ because the sum of the $k$th roots of unity is zero for any $k>1$.  We conclude that $X_j$ is irreducible for $j >0$.  This makes sense, because if $j=0$, then the representation is diagonal and is clearly the direct sum of two other representations: the trivial representation $X^{(1)}$ and the sign representation $X^\tau$ defined by $X^\tau(\tau^e\rho^i) = (-1)^e$.

       We appear to have created $n+1$ irreducible representations (the $n-1$ representations $\{X_j\}_{j=1}^{n-1}$, plus $X^{(1)}$ and $X^\tau$), but 
       these are not all distinct.  It suffices to check which characters are the same, and it is straightforward to see that we
       have $\chi_j(\rho^i) = 2\cos \frac{2\pi i j}{n} = \chi_{n-j}(\rho^i)$,
       and these are the only pairs of characters which are the same.

       That is, for $n$ even, we have given $n/2 + 2$ irreducibles, and for $n$ odd we have given $\lfloor n/2 \rfloor + 2$ 

       We are missing one representation when $n$ is even, which is given by an alternating representation $X^\rho(\tau^e\rho^i) = (-1)^i$.
    \end{enumerate}
    \item[(18)] Suppose that $s_i$ is a transversal for $K \subseteq H$ and $u_i$ is a transversal for $H \subseteq G$.  Then by definition we have
    \[
    ((X\uparrow_K^H)\uparrow_H^G)(g) 
    = \left((X\uparrow_K^H)(u_i^{-1}gu_j)\right)_{i,j}
    = \left(\left(X(s_n^{-1}u_i^{-1}gu_js_m)\right)_{n,m}\right)_{i,j}.
    \]
    By Proposition 1.12.5, it suffices to show that $u_js_i$ is a transversal for $K \subseteq G$ since the representation $X\uparrow_K^G$ does not depend on the choice of transversal.  To see this, suppose we have $u_js_iK = u_ns_mK$.
    Then since $s_i K, s_mK \subseteq H$, we have that the cosets $u_jH$ and $u_nH$ are not disjoint, so they coincide, so $j=n$.  Multiplying by $u_j^{-1}$, we have $s_iK = s_mK$, so immediately we have that $i=m$.  Thus $j=n$ and $i=m$, so we do have a transversal of $K$ in $G$, as desired.
     
\end{enumerate}

\end{document}
